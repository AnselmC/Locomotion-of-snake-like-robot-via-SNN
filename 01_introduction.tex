\chapter{Introduction}
\label{chapter:introduction}

How is the brain able to process vast amounts of information that are needed to recognize a face, for example, so energy efficient?

Although the brain makes up only two percent of the human body weight, it burns up \(20\%\) of the body's energy - around \(20W\) \cite{Drubach1999}.
But compared to modern \acrlong{ann}s (\acrshort{ann}s) this number is quite low.
The most successful networks from the ImageNet Large Scale Visual Recognition Challenge (ILSVRC) \cite{ImageNetWebsite} - a well-known annual competition of \acrshort{ann}s in several visual recognition tasks like AlexNet v2 \cite{Krizhevsky2014} or VGG \cite{Simonyan2014} consume around \(200W\) of CPU and GPU power \cite{Li2016}.
Note that the brain is also capable to do many more tasks than just recognizing images at the same time. \\

Especially when it comes to real-time applications like in autonomous cars or robots where energy efficiency and fast computations are must-haves for a successful deployment, current \acrshort{ann}s are not suitable exactly out of this reasons.
At this point the third generation of \acrshort{ann}s comes into play - \acrlong{snn}s (\acrshort{snn}s) \cite{Ponulak2011}.
Besides the average firing rate of spikes, which is comparable to the neuron value of a conventional \acrshort{ann} they also take into account the precise timing between spikes, resulting in a computational more efficient network as fewer neurons are needed for the same task \cite{Maass1997}.
An ongoing research problem is that the backpropagation algorithm that is utilized in almost every \acrshort{ann} to calculate the weight updates is not applicable to \acrshort{snn}s as the spikes are non differentiable at spike times \cite{Lee2016}.
A promising solution for reinforcement learning like task seems to be the \acrfull{rstdp} learning rule \cite{Florian2007}. \\

In this work, a snake-like robot equipped with a \acrfull{dvs} learns to navigate through a maze using a \acrshort{snn} controller and the \acrfull{rstdp} learning rule.
The \acrshort{snn} controller's ability to cope with different wall heights and maze shapes is demonstrated.
Snake like robots with autonomous locomotion capabilities would be of great use for traversing complex terrain like collapsed factory buildings.
A \acrshort{dvs} has several advantages compared to a normal camera as a pixel of a \acrshort{dvs} only sends data if it perceives a change in light intensity.
Thus, the amount of produced data is greatly reduced, and asynchronous events are created, which suits the character of \acrshort{snn}s perfectly.

In Chapter \ref{chapter:02_background} the background needed to understand \acrshort{snn}s and \acrshort{dvs}s is explained.
Furthermore, the related work is reviewed.
Chapter \ref{chapter:03_methodology} gives details about the methodology of this work by first ...
The training and testing performance of the \acrshort{snn} controller is evaluated in Chapter \ref{chapter:04_discussion}.
Finally in Chapter \ref{chapter:05_conclusion_and_outlook} the work is conclueded and an outlook is given.
